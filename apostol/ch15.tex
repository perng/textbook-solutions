\chapter{Multiple Lebesgue Integrals}


\section{Fubini--Tonelli and Slicing}

\subsection*{Essential Definitions and Theorems}

\begin{definition}[Slicing of Sets]
For a measurable set $S \subset \mathbb{R}^2$, the vertical slice at $x$ is $S^x = \{y : (x, y) \in S\}$ and the horizontal slice at $y$ is $S_y = \{x : (x, y) \in S\}$. These slices represent the cross-sections of $S$ at fixed coordinate values.
\end{definition}

\noindent\textbf{Importance:} Slicing provides a way to understand the structure of sets and functions in higher dimensions by examining their lower-dimensional cross-sections. This is essential for applying Fubini's theorem and for understanding how integrals over regions can be computed as iterated integrals.



\begin{theorem}[Fubini's Theorem]
If $f \in L(\mathbb{R}^2)$, then:
\[\int_{\mathbb{R}^2} f(x, y) \, d(x, y) = \int_{-\infty}^{\infty} \left(\int_{-\infty}^{\infty} f(x, y) \, dy\right) dx = \int_{-\infty}^{\infty} \left(\int_{-\infty}^{\infty} f(x, y) \, dx\right) dy\]
\end{theorem}

\noindent\textbf{Importance:} Fubini's theorem is the fundamental result that allows us to compute double integrals as iterated single integrals. It's essential for practical calculations and provides the theoretical foundation for many applications in physics, probability, and engineering.



\begin{theorem}[Tonelli's Theorem]
If $f$ is nonnegative and measurable on $\mathbb{R}^2$, then:
\[\int_{\mathbb{R}^2} f(x, y) \, d(x, y) = \int_{-\infty}^{\infty} \left(\int_{-\infty}^{\infty} f(x, y) \, dy\right) dx = \int_{-\infty}^{\infty} \left(\int_{-\infty}^{\infty} f(x, y) \, dx\right) dy\]
\end{theorem}

\noindent\textbf{Importance:} Tonelli's theorem provides conditions under which iterated integrals can be interchanged for nonnegative functions. It's particularly useful for proving that functions are integrable and for establishing the validity of Fubini's theorem in specific cases.



\begin{theorem}[Measure of Sliced Sets]
For a measurable set $S \subset \mathbb{R}^2$ with finite measure $\mu(S)$:
\[\mu(S) = \int_{-\infty}^{\infty} \mu(S^x) \, dx = \int_{-\infty}^{\infty} \mu(S_y) \, dy\]
where $\mu(S^x)$ and $\mu(S_y)$ are the measures of the slices.
\end{theorem}

\noindent\textbf{Importance:} This theorem shows how the measure of a two-dimensional set can be computed from the measures of its one-dimensional slices. It's essential for understanding the relationship between measures in different dimensions and for many applications in geometry and analysis.


 


\begin{problembox}[15.1: Integral over Triangular Region]
\begin{problemstatement}
If \( f \in L(T) \), where \( T \) is the triangular region in \( \mathbb{R}^2 \) with vertices at \((0, 0)\), \((1, 0)\), and \((0, 1)\), prove that
\[
\int_T f(x, y) \, d(x, y) = \int_0^1 \left[ \int_0^x f(x, y) \, dy \right] \, dx = \int_0^1 \left[ \int_y^1 f(x, y) \, dx \right] \, dy.
\]
\end{problemstatement}
\end{problembox}

\noindent\textbf{Strategy:} Apply Fubini's Theorem to the indicator function of the triangular region. Express the double integral as an iterated integral using two different slicing methods: first by vertical lines (fixing x and integrating over y), then by horizontal lines (fixing y and integrating over x).

\bigskip\noindent\textbf{Solution:}
Write the indicator of the triangle \(T=\{(x,y): 0\le y \le x \le 1\}\) and apply Fubini to
\[
\int_{\mathbb{R}^2} f(x,y) \mathbf{1}_T(x,y)\, d(x,y).
\]
Slicing by vertical lines gives \(\int_0^1\!\big[\int_0^x f(x,y)\,dy\big]dx\). Slicing by horizontal lines gives \(\int_0^1\!\big[\int_y^1 f(x,y)\,dx\big]dy\). Hence the displayed equalities.\qed


\begin{problembox}[15.2: Double Integral Calculation]
\begin{problemstatement}
For fixed \( c, 0 < c < 1 \), define \( f \) on \( \mathbb{R}^2 \) as follows:
\[
f(x, y) = 
\begin{cases} 
(1 - y)^c / (x - y)^c & \text{if } 0 \leq y < x, 0 < x < 1, \\
0 & \text{otherwise}.
\end{cases}
\]
Prove that \( f \in L(\mathbb{R}^2) \) and calculate the double integral 
\[
\int_{\mathbb{R}^2} f(x, y) \, d(x, y).
\]
\end{problemstatement}
\end{problembox}

\noindent\textbf{Strategy:} First identify the support of the function as a triangular region. Then use a change of variables \(u = x-y\), \(v = y\) to simplify the integrand and transform the region to a more manageable shape. The Jacobian is 1, making the transformation straightforward.

\bigskip\noindent\textbf{Solution:}
The support is the triangle \(0\le y < x < 1\), so
\[
\int_{\mathbb{R}^2} f = \int_0^1\!\int_0^x \frac{(1-y)^c}{(x-y)^c}\,dy\,dx.
\]
Let \(u=x-y\), \(v=y\). The Jacobian is 1 and the region is \(0<v<1\), \(0<u<1-v\). Then
\[
\int_0^1\!\int_0^{1-v} u^{-c}(1-v)^c\,du\,dv
= \frac{1}{1-c}\int_0^1 (1-v)\,dv = \frac{1}{2(1-c)}.
\]
Thus \(f\in L(\mathbb{R}^2)\) for \(0<c<1\) and the value is \(1/[2(1-c)]\).\qed


\begin{problembox}[15.3: Measure of a Subset]
\begin{problemstatement}
Let \( S \) be a measurable subset of \( \mathbb{R}^2 \) with finite measure \( \mu(S) \). Using the notation of Definition 15.4, prove that
\[
\mu(S) = \int_{-\infty}^{\infty} \mu(S^x) \, dx = \int_{-\infty}^{\infty} \mu(S_y) \, dy.
\]
\end{problemstatement}
\end{problembox}

\noindent\textbf{Strategy:} Apply Fubini's Theorem to the indicator function \(\mathbf{1}_S\). Use the definition of the sliced sets \(S^x\) and \(S_y\) from Definition 15.4, which represent the cross-sections of \(S\) at fixed \(x\) and \(y\) values respectively.

\bigskip\noindent\textbf{Solution:}
Apply Fubini to the indicator function \(\mathbf{1}_S\). By Definition 15.4, \(\mu(S^x)=\int \mathbf{1}_S(x,y)\,dy\) and \(\mu(S_y)=\int \mathbf{1}_S(x,y)\,dx\). Hence
\[
\mu(S)=\iint \mathbf{1}_S\,d(x,y)=\int_{-\infty}^{\infty}\!\mu(S^x)\,dx=\int_{-\infty}^{\infty}\!\mu(S_y)\,dy.
\]\qed


\begin{problembox}[15.4: Iterated Integrals vs Double Integral]
\begin{problemstatement}
Let \( f(x, y) = e^{-xy} \sin x \sin y \) if \( x \geq 0, y \geq 0 \), and let \( f(x, y) = 0 \) otherwise. Prove that both iterated integrals
\[
\int_{\mathbb{R}^1} \left[ \int_{\mathbb{R}^1} f(x, y) \, dx \right] \, dy \quad \text{and} \quad \int_{\mathbb{R}^1} \left[ \int_{\mathbb{R}^1} f(x, y) \, dy \right] \, dx
\]
exist and are equal, but that the double integral of \( f \) over \( \mathbb{R}^2 \) does not exist. Also, explain why this does not contradict the Tonelli-Hobson test (Theorem 15.8).
\end{problemstatement}
\end{problembox}

\noindent\textbf{Strategy:} For the iterated integrals, use the Laplace transform formula for \(\int_0^\infty e^{-at} \sin t \, dt\). Show that both iterated integrals converge absolutely. For the double integral, demonstrate that \(|f|\) is not integrable by showing divergence near the axes. Explain why Tonelli's theorem doesn't apply since \(f\) changes sign and is not absolutely integrable.

\bigskip\noindent\textbf{Solution:}
Since \(f\) is supported in the first quadrant, we may write the iterated integrals as
\[
\int_0^\infty \left[ \int_0^\infty e^{-xy} \sin x \sin y \, dx \right] \, dy,
\quad
\int_0^\infty \left[ \int_0^\infty e^{-xy} \sin x \sin y \, dy \right] \, dx.
\]

\emph{First iterated integral.}
For fixed \(y \ge 0\),
\[
\int_0^\infty e^{-xy} \sin x \, dx
= \frac{1}{y^2+1},
\]
by the standard Laplace transform formula 
\(\int_0^\infty e^{-at} \sin t\, dt = \frac{1}{a^2+1}\) for \(a > -1\).
Thus
\[
\int_0^\infty e^{-xy} \sin x \sin y \, dx
= \sin y \cdot \frac{1}{y^2+1}.
\]
The outer integral becomes
\[
\int_0^\infty \frac{\sin y}{y^2+1} \, dy,
\]
which converges absolutely since \(\frac{|\sin y|}{y^2+1} \le \frac{1}{y^2+1}\) and \(\int_0^\infty \frac{dy}{y^2+1} < \infty\).

\emph{Second iterated integral.}
For fixed \(x \ge 0\),
\[
\int_0^\infty e^{-xy} \sin y \, dy
= \frac{1}{x^2+1},
\]
again by the Laplace transform formula (now with roles of \(x\) and \(y\) interchanged).
Thus
\[
\int_0^\infty e^{-xy} \sin x \sin y \, dy
= \sin x \cdot \frac{1}{x^2+1}.
\]
The outer integral becomes
\[
\int_0^\infty \frac{\sin x}{x^2+1} \, dx,
\]
which converges absolutely by the same comparison as above.  

Therefore both iterated integrals exist and
\[
\int_{\mathbb{R}^1} \left[ \int_{\mathbb{R}^1} f(x, y) \, dx \right] \, dy
=
\int_{\mathbb{R}^1} \left[ \int_{\mathbb{R}^1} f(x, y) \, dy \right] \, dx
= \int_0^\infty \frac{\sin t}{t^2+1} \, dt.
\]

\emph{Non-existence of the double integral.}
The double integral
\[
\iint_{\mathbb{R}^2} f(x,y) \, dx \, dy
\]
in the Lebesgue sense exists only if \(f\) is absolutely integrable:
\[
\iint_{\mathbb{R}^2} |f(x,y)| \, dx \, dy < \infty.
\]
But for \(x,y \ge 0\),
\[
|f(x,y)| = e^{-xy} |\sin x| \, |\sin y| \ge 0.
\]
Separate variables:
\[
\iint_{[0,\infty)^2} e^{-xy} |\sin x|\, |\sin y| \, dx\, dy
\]
fails to converge. Indeed, near the \(x\)-axis (\(y \to 0^+\)), \(e^{-xy} \approx 1\) and the inner \(x\)-integral over \([0,\infty)\) of \(|\sin x|\) diverges, giving divergence of the absolute integral.  
Hence \(f \notin L^1(\mathbb{R}^2)\) and the double integral does not exist in the Lebesgue sense.

\emph{Why no contradiction to Tonelli--Hobson.}
Tonelli's theorem applies to nonnegative functions and ensures that if \(\iint |f| < \infty\) then Fubini's theorem allows exchanging order of integration.  
Here \(f\) changes sign and is not absolutely integrable, so Tonelli's theorem does not apply.  
Hobson's test also requires certain boundedness conditions on one of the iterated integrals of \(|f|\), which fail here because \(\int |f(x,y)|\, dx = \infty\) for every small \(y > 0\).  
Therefore there is no contradiction: both iterated integrals exist and are equal, yet the double integral does not exist because \(f \notin L^1(\mathbb{R}^2)\).\qed
\section{Non-Integrable Examples and Iterated Integrals}

\subsection*{Essential Definitions and Theorems}

\begin{definition}[Lebesgue Integrability]
A function $f$ is Lebesgue integrable on a measurable set $S$ if both $f^+$ and $f^-$ are integrable, where $f^+ = \max(f, 0)$ and $f^- = \max(-f, 0)$. Equivalently, $f$ is Lebesgue integrable if $|f|$ is integrable.
\end{definition}

\noindent\textbf{Importance:} Lebesgue integrability provides a more robust framework than Riemann integration, allowing integration of a wider class of functions. The requirement that $|f|$ be integrable ensures that the integral is well-defined and finite.



\begin{theorem}[Criterion for Non-Integrability]
If both iterated integrals of a function $f$ over a rectangle exist but are unequal, then $f$ is not Lebesgue integrable on that rectangle.
\end{theorem}

\noindent\textbf{Importance:} This criterion provides a practical way to show that functions are not integrable by computing their iterated integrals. It's particularly useful for functions that have singularities or oscillatory behavior that prevents absolute integrability.



\begin{theorem}[Comparison with Improper Integrals]
If a function $f$ is not absolutely integrable, then it may have iterated integrals that exist as improper integrals but the function is not Lebesgue integrable.
\end{theorem}

\noindent\textbf{Importance:} This theorem explains why some functions can have well-defined iterated integrals (as limits of proper integrals) but fail to be Lebesgue integrable. This distinction is crucial for understanding the limitations of different integration theories.



\begin{theorem}[Fubini's Theorem for Lebesgue Integrals]
If $f$ is Lebesgue integrable on $\mathbb{R}^2$, then both iterated integrals exist and are equal to the double integral. However, the existence of iterated integrals does not guarantee Lebesgue integrability.
\end{theorem}

\noindent\textbf{Importance:} This theorem shows that while Fubini's theorem provides a powerful tool for computing integrals, the existence of iterated integrals alone is not sufficient to guarantee integrability. This is essential for understanding the relationship between different types of integration.





\begin{problembox}[15.5: Non-Integrable Function]
\begin{problemstatement}
Let \( f(x, y) = (x^2 - y^2)/(x^2 + y^2)^2 \) for \( 0 \leq x \leq 1, 0 < y \leq 1 \), and let \( f(0, 0) = 0 \). Prove that both iterated integrals
\[
\int_0^1 \left[ \int_0^1 f(x, y) \, dy \right] \, dx \quad \text{and} \quad \int_0^1 \left[ \int_0^1 f(x, y) \, dx \right] \, dy
\]
exist but are not equal. This shows that \( f \) is not Lebesgue-integrable on \([0, 1] \times [0, 1]\).
\end{problemstatement}
\end{problembox}

\noindent\textbf{Strategy:} Compute both iterated integrals directly using partial fraction decomposition or substitution. Show they yield different values (\(\pi/4\) and \(-\pi/4\)), which by the criterion that unequal iterated integrals imply non-integrability, proves \(f \notin L([0,1]^2)\).

\bigskip\noindent\textbf{Solution:}
For fixed \(x>0\),
\[
\int_0^1 \frac{x^2-y^2}{(x^2+y^2)^2}\,dy = \frac{1}{1+x^2}.
\]
Thus \(\int_0^1[\int_0^1 f(x,y)\,dy]dx=\int_0^1\!\frac{dx}{1+x^2}=\pi/4\). For fixed \(y>0\), symmetry gives
\[
\int_0^1 \frac{x^2-y^2}{(x^2+y^2)^2}\,dx = -\frac{1}{1+y^2},
\]
so \(\int_0^1[\int_0^1 f(x,y)\,dx]dy=-\pi/4\). Since the iterated integrals exist but are unequal, \(f\notin L([0,1]^2)\).\qed


\begin{problembox}[15.6: Another Non-Integrable Function]
\begin{problemstatement}
Let \( I = [0, 1] \times [0, 1] \), let \( f(x, y) = (x - y)/(x + y)^3 \) if \( (x, y) \in I \), \( (x, y) \neq (0, 0) \), and let \( f(0, 0) = 0 \). Prove that \( f \notin L(I) \) by considering the iterated integrals
\[
\int_0^1 \left[ \int_0^1 f(x, y) \, dy \right] \, dx \quad \text{and} \quad \int_0^1 \left[ \int_0^1 f(x, y) \, dx \right] \, dy.
\]
\end{problemstatement}
\end{problembox}

\noindent\textbf{Strategy:} Similar to the previous problem, compute both iterated integrals directly. Use the symmetry of the function to show they give opposite values (\(1/2\) and \(-1/2\)), establishing non-integrability by the same criterion.

\bigskip\noindent\textbf{Solution:}
For fixed \(x\in[0,1]\),
\[
\int_0^1 \frac{x-y}{(x+y)^3}\,dy = \frac{1}{(1+x)^{2}},
\]
so the outer \(x\)-integral equals \(\int_0^1 (1+x)^{-2}dx=1/2\). For fixed \(y\in[0,1]\),
\[
\int_0^1 \frac{x-y}{(x+y)^3}\,dx = -\frac{1}{(1+y)^{2}},
\]
so the other iterated integral equals \(-1/2\). Hence \(f\notin L(I)\).\qed


\begin{problembox}[15.7: Non-Integrable Function on Infinite Interval]
\begin{problemstatement}
Let \( I = [0, 1] \times [1, +\infty) \) and let \( f(x, y) = e^{-xy} - 2e^{-2xy} \) if \( (x, y) \in I \). Prove that \( f \notin L(I) \) by considering the iterated integrals
\[
\int_0^1 \left[ \int_0^\infty f(x, y) \, dy \right] \, dx \quad \text{and} \quad \int_1^\infty \left[ \int_0^1 f(x, y) \, dx \right] \, dy.
\]
\end{problemstatement}
\end{problembox}

\noindent\textbf{Strategy:} Show that one iterated integral diverges while the other converges. The first integral diverges due to a singularity at \(x=0\), while the second converges absolutely. This demonstrates non-integrability since both iterated integrals must converge for Lebesgue integrability.

\bigskip\noindent\textbf{Solution:}
For \(x\in(0,1]\),
\[
\int_1^{\infty} \big(e^{-xy}-2e^{-2xy}\big)\,dy = \frac{e^{-x}-2e^{-2x}}{x},
\]
whose integral in \(x\in(0,1]\) diverges at 0, so \(\int_0^1[\int_1^{\infty} f\,dy]dx\) diverges. On the other hand, for fixed \(y\ge 1\),
\[
\int_0^1 \big(e^{-xy}-2e^{-2xy}\big)\,dx = \frac{e^{-2y}-e^{-y}}{y},
\]
and \(\int_1^{\infty} |e^{-2y}-e^{-y}|\,y^{-1} dy<\infty\). Thus one iterated integral converges while the other diverges, so \(f\notin L(I)\).\qed
\section{Change of Variables}

\subsection*{Essential Definitions and Theorems}

\begin{definition}[Change of Variables Theorem]
If $T: \mathbb{R}^n \to \mathbb{R}^n$ is a one-to-one $C^1$ transformation with nonvanishing Jacobian determinant on a measurable set $S$, and $f$ is integrable on $T(S)$, then:
\[\int_{T(S)} f(x) \, dx = \int_S f(T(u)) |J_T(u)| \, du\]
where $J_T(u)$ is the Jacobian determinant of $T$ at $u$.
\end{definition}

\noindent\textbf{Importance:} The Change of Variables Theorem is the fundamental tool for computing integrals in different coordinate systems. It allows us to transform complicated regions into simpler ones and to use coordinate systems that are natural for the problem at hand. The Jacobian determinant accounts for how the transformation changes volumes locally.



\begin{definition}[Jacobian Determinant]
For a $C^1$ transformation $T: \mathbb{R}^n \to \mathbb{R}^n$ given by $T(u_1, \ldots, u_n) = (x_1, \ldots, x_n)$, the Jacobian determinant is:
\[J_T(u) = \det\left[\frac{\partial x_i}{\partial u_j}\right] = \begin{vmatrix}
\frac{\partial x_1}{\partial u_1} & \cdots & \frac{\partial x_1}{\partial u_n} \\
\vdots & \ddots & \vdots \\
\frac{\partial x_n}{\partial u_1} & \cdots & \frac{\partial x_n}{\partial u_n}
\end{vmatrix}\]
\end{definition}

\noindent\textbf{Importance:} The Jacobian determinant measures how the transformation $T$ changes volumes locally. Its absolute value gives the scaling factor for volume elements, and its sign indicates whether the transformation preserves or reverses orientation. This is essential for understanding how integrals transform under coordinate changes.



\begin{theorem}[Polar Coordinates Transformation]
The transformation from polar to Cartesian coordinates $(r, \theta) \mapsto (x, y) = (r\cos\theta, r\sin\theta)$ has Jacobian determinant $J = r$. Therefore:
\[\iint_D f(x, y) \, dx \, dy = \iint_{D'} f(r\cos\theta, r\sin\theta) \, r \, dr \, d\theta\]
\end{theorem}

\noindent\textbf{Importance:} Polar coordinates are essential for problems with circular or radial symmetry. The factor $r$ in the Jacobian accounts for the fact that area elements in polar coordinates scale with distance from the origin. This transformation is fundamental for many applications in physics and engineering.



\begin{theorem}[Spherical Coordinates Transformation]
The transformation from spherical to Cartesian coordinates $(\rho, \theta, \varphi) \mapsto (x, y, z) = (\rho\cos\theta\sin\varphi, \rho\sin\theta\sin\varphi, \rho\cos\varphi)$ has Jacobian determinant $J = \rho^2\sin\varphi$. Therefore:
\[\iiint_D f(x, y, z) \, dx \, dy \, dz = \iiint_{D'} f(\rho\cos\theta\sin\varphi, \rho\sin\theta\sin\varphi, \rho\cos\varphi) \, \rho^2\sin\varphi \, d\rho \, d\theta \, d\varphi\]
\end{theorem}

\noindent\textbf{Importance:} Spherical coordinates are essential for problems with spherical symmetry, such as gravitational fields, electromagnetic fields, and quantum mechanics. The factor $\rho^2\sin\varphi$ accounts for the volume scaling in spherical coordinates and is crucial for correct integration.





\begin{problembox}[15.8: Transformation of Integrals]
\begin{problemstatement}
The following formulas for transforming double and triple integrals occur in elementary calculus. Obtain them as consequences of Theorem 15.11 and give restrictions on \( T \) and \( T' \) for validity of these formulas.
\begin{enumerate}[label=(\alph*)]
\item \[ \iint_T f(x, y) \, dx \, dy = \iint_{T'} f(r \cos \theta, r \sin \theta)r \, dr \, d\theta.\]
\item \[ \iiint_T f(x, y, z) \, dx \, dy \, dz = \iiint_{T'} f(r \cos \theta, r \sin \theta, z)r \, dr \, d\theta \, dz.\]
\item 
\begin{align*}
&\iiint_T f(x, y, z) \, dx \, dy \, dz \\
=& \iiint_{T'} f(\rho \cos \theta \sin \varphi, \rho \sin \theta \sin \varphi, \rho \cos \varphi) \rho^2 \sin \varphi \, d\rho \, d\theta \, d\varphi.
\end{align*}
\end{enumerate}
\end{problemstatement}
\end{problembox}

\noindent\textbf{Strategy:} Apply the Change of Variables Theorem (Theorem 15.11) to the coordinate transformations for polar, cylindrical, and spherical coordinates. Calculate the Jacobian determinants for each transformation and specify the conditions on \(T\) and \(T'\) for the theorem to apply.

\bigskip\noindent\textbf{Solution:}
These follow from the Change of Variables Theorem applied to the maps
\( (r,\theta)\mapsto (r\cos\theta, r\sin\theta) \),
\( (r,\theta,z)\mapsto (r\cos\theta, r\sin\theta, z) \), and
\( (\rho,\theta,\varphi)\mapsto (\rho\cos\theta\sin\varphi,\rho\sin\theta\sin\varphi,\rho\cos\varphi) \).
The determinants are respectively \(r\), \(r\), and \(\rho^2\sin\varphi\). Validity requires that \(T'\) be a measurable set on which the coordinate map is one-to-one (modulo negligible sets), with measurable inverse onto \(T\); typically take \(T'\) a rectangle in coordinates with appropriate ranges and \(T\) the corresponding polar/cylindrical/spherical image.\qed
\section{Gaussian Integrals}

\subsection*{Essential Definitions and Theorems}

\begin{definition}[Gaussian Function]
A Gaussian function (or normal distribution) is a function of the form:
\[f(x) = \frac{1}{\sigma\sqrt{2\pi}} e^{-\frac{(x-\mu)^2}{2\sigma^2}}\]
where $\mu$ is the mean and $\sigma$ is the standard deviation. The standard Gaussian has $\mu = 0$ and $\sigma = 1$.
\end{definition}

\noindent\textbf{Importance:} Gaussian functions are fundamental in probability theory, statistics, and many areas of physics. They represent the most common distribution for random variables and appear naturally in many physical processes due to the central limit theorem. The Gaussian shape is characterized by its bell curve and exponential decay.



\begin{definition}[Gamma Function]
The Gamma function is defined for $\text{Re}(z) > 0$ by:
\[\Gamma(z) = \int_0^{\infty} t^{z-1} e^{-t} \, dt\]
It satisfies $\Gamma(n+1) = n!$ for positive integers $n$ and $\Gamma(1/2) = \sqrt{\pi}$.
\end{definition}

\noindent\textbf{Importance:} The Gamma function is a generalization of the factorial function to complex numbers. It appears naturally in many integrals involving exponentials and powers, particularly in probability theory and physics. The special values $\Gamma(1/2) = \sqrt{\pi}$ and $\Gamma(n+1) = n!$ are crucial for computing Gaussian integrals.



\begin{theorem}[Basic Gaussian Integral]
The fundamental Gaussian integral is:
\[\int_{-\infty}^{\infty} e^{-x^2} \, dx = \sqrt{\pi}\]
\end{theorem}

\noindent\textbf{Importance:} This is the most fundamental Gaussian integral, from which all other Gaussian integrals can be derived. It's essential for probability theory, as it normalizes the standard normal distribution. The fact that this integral equals $\sqrt{\pi}$ is surprising and beautiful, connecting analysis with geometry.



\begin{theorem}[Product Structure of Gaussian Measures]
If $f(x) = e^{-x^2}$ and $g(y) = e^{-y^2}$, then:
\[\int_{\mathbb{R}^2} e^{-(x^2 + y^2)} \, d(x, y) = \left(\int_{-\infty}^{\infty} e^{-x^2} \, dx\right) \left(\int_{-\infty}^{\infty} e^{-y^2} \, dy\right) = \pi\]
This extends to higher dimensions: $\int_{\mathbb{R}^n} e^{-\|x\|^2} \, dx = \pi^{n/2}$.
\end{theorem}

\noindent\textbf{Importance:} This theorem shows that Gaussian measures in higher dimensions factorize into products of one-dimensional Gaussian measures. This property is crucial for computing multivariate normal distributions and for understanding the geometry of Gaussian functions in higher dimensions.



\begin{theorem}[Scaling Property of Gaussian Integrals]
For $t > 0$:
\[\int_{-\infty}^{\infty} e^{-tx^2} \, dx = \sqrt{\frac{\pi}{t}}\]
and by differentiation:
\[\int_{-\infty}^{\infty} x^2 e^{-tx^2} \, dx = \frac{\sqrt{\pi}}{2} t^{-3/2}\]
\end{theorem}

\noindent\textbf{Importance:} The scaling property allows us to compute Gaussian integrals with different parameters from the basic integral. This is essential for computing moments of normal distributions and for applications in physics where the temperature or energy scale varies. The differentiation technique is a powerful method for computing higher moments.





\begin{problembox}[15.9: Gaussian Integrals]
\begin{problemstatement}
\begin{enumerate}[label=(\alph*)]
\item Prove that \(\int_{\mathbb{R}^2} e^{-(x^2 + y^2)} \, d(x, y) = \pi\) by transforming the integral to polar coordinates.
\item Use part (a) to prove that \(\int_{-\infty}^{\infty} e^{-x^2} \, dx = \sqrt{\pi}\).
\item Use part (b) to prove that \(\int_{\mathbb{R}^n} e^{-\|x\|^2} \, d(x_1, \ldots, x_n) = \pi^{n/2}\).
\item Use part (b) to calculate \(\int_{-\infty}^{\infty} e^{-tx^2} \, dx\) and \(\int_{-\infty}^{\infty} x^2 e^{-tx^2} \, dx, t > 0\).
\end{enumerate}
\end{problemstatement}
\end{problembox}

\noindent\textbf{Strategy:} Start with polar coordinates for the 2D integral, then use the product structure of Gaussian measures to extend to higher dimensions. For the parameterized integrals, use scaling and differentiation techniques to derive the results from the basic Gaussian integral.

\bigskip\noindent\textbf{Solution:}
\begin{enumerate}[label=(\alph*)]
\item In polar coordinates,
\(\int_{\mathbb{R}^2} e^{-(x^2+y^2)} d(x,y)=\int_0^{2\pi}\int_0^{\infty} e^{-r^2} r\,dr\,d\theta=2\pi\cdot\tfrac12=\pi\).
\item By symmetry, \(\left(\int_{-\infty}^{\infty} e^{-x^2}dx\right)^2=\int_{\mathbb{R}^2} e^{-(x^2+y^2)}d(x,y)=\pi\). Hence the one-dimensional integral equals \(\sqrt{\pi}\).
\item Using product structure, \(\int_{\mathbb{R}^n} e^{-\|x\|^2}dx=\big(\int_{-\infty}^{\infty} e^{-t^2}dt\big)^n=\pi^{n/2}\).
\item Scaling gives \(\int_{-\infty}^{\infty} e^{-t x^2} dx = t^{-1/2}\int e^{-u^2}du = \sqrt{\pi/t}\). Differentiating in \(t\) or using symmetry yields \(\int x^2 e^{-t x^2} dx = \tfrac{\sqrt{\pi}}{2} t^{-3/2}\).
\end{enumerate}\qed
\section{Volumes of n-Balls}

\subsection*{Essential Definitions and Theorems}

\begin{definition}[n-Ball]
The $n$-ball of radius $a$ centered at the origin is the set:
\[B(0; a) = \{x \in \mathbb{R}^n : \|x\| \leq a\}\]
where $\|x\| = \sqrt{x_1^2 + \cdots + x_n^2}$ is the Euclidean norm.
\end{definition}

\noindent\textbf{Importance:} n-balls are fundamental geometric objects that generalize circles and spheres to higher dimensions. They appear naturally in many areas of mathematics and physics, from geometry and analysis to probability theory and statistical mechanics. Understanding their volumes is essential for many applications.



\begin{definition}[Volume Scaling]
For any measurable set $S \subset \mathbb{R}^n$ and linear transformation $T(x) = Ax$, the volume scales as:
\[\text{vol}(T(S)) = |\det(A)| \cdot \text{vol}(S)\]
\end{definition}

\noindent\textbf{Importance:} Volume scaling is a fundamental property that shows how volumes change under linear transformations. For n-balls, this means that $V_n(a) = a^n V_n(1)$, which greatly simplifies volume calculations. This scaling property is essential for understanding geometric transformations and their effects on volumes.



\begin{theorem}[Volume of n-Ball]
The volume of the n-ball of radius $a$ is:
\[V_n(a) = \frac{\pi^{n/2} a^n}{\Gamma(\frac{n}{2} + 1)}\]
where $\Gamma$ is the Gamma function.
\end{theorem}

\noindent\textbf{Importance:} This is the fundamental formula for computing volumes of n-balls. It connects geometry with analysis through the Gamma function and provides a unified expression for all dimensions. The formula shows how volumes grow with dimension and radius, which is crucial for understanding high-dimensional geometry.



\begin{theorem}[Recursion Formula for n-Ball Volume]
For $n \geq 3$, the volume of the unit n-ball satisfies:
\[V_n(1) = V_{n-2}(1) \cdot \frac{2\pi}{n}\]
\end{theorem}

\noindent\textbf{Importance:} This recursion formula provides an efficient way to compute n-ball volumes by reducing the problem to lower dimensions. It shows the relationship between volumes in different dimensions and is essential for understanding how volumes behave as dimension increases. The factor $2\pi/n$ reflects the geometric relationship between dimensions.



\begin{theorem}[Surface Area of n-Sphere]
The surface area of the unit n-sphere is:
\[\omega_n = n V_n(1) = \frac{2\pi^{n/2}}{\Gamma(\frac{n}{2})}\]
\end{theorem}

\noindent\textbf{Importance:} The surface area of n-spheres is closely related to the volume of n-balls and appears in many applications, particularly in physics and probability theory. The relationship $\omega_n = n V_n(1)$ shows how surface area and volume are connected in higher dimensions.





\begin{problembox}[15.10: Volume of \( n \)-Ball]
\begin{problemstatement}
Let \( V_n(a) \) denote the \( n \)-measure of the \( n \)-ball \( B(0; a) \) of radius \( a \). This exercise outlines a proof of the formula
\[
V_n(a) = \frac{\pi^{n/2} a^n}{\Gamma( \frac{1}{2} n + 1 )}.
\]
\begin{enumerate}[label=(\alph*)]
\item Use a linear change of variable to prove that \( V_n(a) = a^n V_n(1) \).
\item Assume \( n \geq 3 \), express the integral for \( V_n(1) \) as the iteration of an \( (n - 2) \)-fold integral and a double integral, and use part (a) for an \( (n - 2) \)-ball to obtain the formula
\[
V_n(1) = V_{n-2}(1) \int_0^{2\pi} \left[ \int_0^1 (1 - r^2)^{n/2 - 1}r \, dr \right] d\theta = V_{n-2}(1) \frac{2\pi}{n}.
\]
\item From the recursion formula in (b) deduce that
\[
V_n(1) = \frac{\pi^{n/2}}{\Gamma(\frac{1}{2}n + 1)}.
\]
\end{enumerate}
\end{problemstatement}
\end{problembox}

\noindent\textbf{Strategy:} Use scaling properties for part (a). For part (b), decompose the \(n\)-ball into a product of a 2D disk and an \((n-2)\)-ball, then use polar coordinates for the 2D part. For part (c), use mathematical induction with the recursion formula and known values for low dimensions.

\bigskip\noindent\textbf{Solution:}
\begin{enumerate}[label=(\alph*)]
\item Under \(x\mapsto ax\), \(n\)-volume scales by \(a^n\), so \(V_n(a)=a^n V_n(1)\).
\item For \(n\ge 3\), write \(\|x\|^2=r^2+\rho^2\) with \(\rho\in\mathbb{R}^{n-2}\). Then
\[
V_n(1)=V_{n-2}(1)\int_0^{2\pi}\!\int_0^1 (1-r^2)^{\frac{n}{2}-1} r\,dr\,d\theta = V_{n-2}(1)\cdot \frac{2\pi}{n}.
\]
\item Induct on \(n\) using (b). With \(V_0(1)=1\), \(V_1(1)=2\), the recursion yields \(V_n(1)=\dfrac{\pi^{n/2}}{\Gamma(\tfrac{n}{2}+1)}\).
\end{enumerate}\qed


\begin{problembox}[15.11: Integral over \( n \)-Ball]
\begin{problemstatement}
Refer to Exercise 15.10 and prove that
\[
\int_{B(0;1)} x_k^2 \, d(x_1, \ldots, x_n) = \frac{V_n(1)}{n + 2}
\]
for each \( k = 1, 2, \ldots, n \).
\end{problemstatement}
\end{problembox}

\noindent\textbf{Strategy:} Use the symmetry of the ball to show all \(\int x_k^2\) are equal. Then use the identity \(\sum_{k=1}^n x_k^2 = \|x\|^2\) and spherical coordinates to compute \(\int \|x\|^2\) in terms of the surface area of the unit sphere.

\bigskip\noindent\textbf{Solution:}
By symmetry, \(\int_{B(0;1)} x_k^2\,dx\) is the same for each \(k\) and \(\sum_{k=1}^n x_k^2=\|x\|^2\). Hence
\[
\sum_{k=1}^n \int_{B(0;1)} x_k^2\,dx = \int_{B(0;1)} \|x\|^2\,dx.
\]
Using spherical coordinates, \(\int_{B(0;1)} \|x\|^2 dx = \omega_{n}\int_0^1 r^{n+1}dr=\omega_n/(n+2)\), where \(\omega_n= n V_n(1)\) is the surface area of the unit sphere. Therefore each \(\int x_k^2 = V_n(1)/(n+2)\).\qed


\begin{problembox}[15.12: Recursion Formula for \( n \)-Ball Volume]
\begin{problemstatement}
Refer to Exercise 15.10 and express the integral for \( V_n(1) \) as the iteration of an \( (n - 1) \)-fold integral and a one-dimensional integral, to obtain the recursion formula
\[
V_n(1) = 2V_{n-1}(1) \int_0^1 (1 - x^2)^{(n-1)/2} \, dx.
\]
Put \( x = \cos t \) in the integral, and use the formula of Exercise 15.10 to deduce that
\[
\int_0^{\pi/2} \cos^n t \, dt = \frac{\sqrt{\pi}}{2} \frac{\Gamma(\frac{1}{2}n + \frac{1}{2})}{\Gamma(\frac{1}{2}n + 1)}.
\]
\end{problemstatement}
\end{problembox}

\noindent\textbf{Strategy:} Decompose the \(n\)-ball into a product of a 1D interval and an \((n-1)\)-ball, then use the substitution \(x = \cos t\) to transform the integral. Combine this with the result from Exercise 15.10 to derive the cosine integral formula.

\bigskip\noindent\textbf{Solution:}
Writing \(V_n(1)=2V_{n-1}(1)\int_0^1 (1-x^2)^{(n-1)/2}dx\) and substituting \(x=\cos t\) gives
\[
\int_0^{\pi/2} \cos^n t\,dt = \frac{\sqrt{\pi}}{2}\,\frac{\Gamma(\tfrac{n+1}{2})}{\Gamma(\tfrac{n}{2}+1)}.
\]
Combining with Exercise 15.10 yields the stated recursion.\qed
\section{Volumes in Other Regions}

\subsection*{Essential Definitions and Theorems}

\begin{definition}[n-Dimensional Diamond]
The $n$-dimensional diamond (or cross-polytope) of radius $a$ is the set:
\[S_n(a) = \{(x_1, \ldots, x_n) : |x_1| + \cdots + |x_n| \leq a\}\]
\end{definition}

\noindent\textbf{Importance:} The n-dimensional diamond is a geometric object that generalizes the diamond shape to higher dimensions and appears in optimization.

\begin{definition}[Simplex]
An $n$-dimensional simplex is the convex hull of $n+1$ affinely independent points. The standard simplex is:
\[\Delta_n = \{(x_1, \ldots, x_n) : x_i \geq 0, x_1 + \cdots + x_n \leq 1\}\]
\end{definition}

\noindent\textbf{Importance:} Simplices are the simplest convex polytopes and serve as building blocks for more complex geometric objects.

\begin{theorem}[Volume of n-Dimensional Diamond]
The volume of the n-dimensional diamond of radius $a$ is:
\[V_n(a) = \frac{2^n a^n}{n!}\]
\end{theorem}

\noindent\textbf{Importance:} This formula shows that the volume of the n-dimensional diamond grows factorially with dimension, which is much faster than the exponential growth of n-ball volumes. This difference is crucial for understanding the geometry of high-dimensional spaces and has important implications for optimization and probability theory.



\begin{theorem}[Volume of Standard Simplex]
The volume of the standard n-dimensional simplex is:
\[\text{vol}(\Delta_n) = \frac{1}{n!}\]
\end{theorem}

\noindent\textbf{Importance:} This simple formula for the volume of the standard simplex is fundamental for many applications. It shows that simplex volumes decrease factorially with dimension, which is important for understanding the geometry of high-dimensional spaces and for numerical methods that use simplices.



\begin{theorem}[Slicing Method for Volume Calculation]
If a region $S \subset \mathbb{R}^n$ can be sliced by fixing one coordinate, then:
\[\text{vol}(S) = \int_a^b \text{vol}_{n-1}(S_x) \, dx\]
where $S_x$ is the $(n-1)$-dimensional cross-section at coordinate value $x$.
\end{theorem}

\noindent\textbf{Importance:} The slicing method is a powerful technique for computing volumes of complex regions by reducing them to lower-dimensional problems. It's particularly useful for regions with symmetry or special structure, and provides a systematic way to approach volume calculations in higher dimensions.





\begin{problembox}[15.13: Volume of \( n \)-Dimensional Diamond]
\begin{problemstatement}
If \( a > 0 \), let \( S_n(a) = \{(x_1, \ldots, x_n): |x_1| + \cdots + |x_n| \leq a\} \), and let \( V_n(a) \) denote the \( n \)-measure of \( S_n(a) \). This exercise outlines a proof of the formula \( V_n(a) = 2^n a^n / n! \).
\begin{enumerate}[label=(\alph*)]
\item Use a linear change of variable to prove that \( V_n(a) = a^n V_n(1) \).
\item Assume \( n \geq 2 \), express the integral for \( V_n(1) \) as an iteration of a one-dimensional integral and an \( (n - 1) \)-fold integral, use (a) to show that
\[
V_n(1) = V_{n-1}(1) \int_{-1}^1 (1 - |x|)^{n-1} \, dx = 2V_{n-1}(1)/n,
\]
and deduce that \( V_n(1) = 2^n / n! \).
\end{enumerate}
\end{problemstatement}
\end{problembox}

\noindent\textbf{Strategy:} Use scaling for part (a). For part (b), slice the diamond by fixing one coordinate and use the fact that the cross-section is a scaled \((n-1)\)-dimensional diamond. Use mathematical induction to establish the factorial formula.

\bigskip\noindent\textbf{Solution:}
\begin{enumerate}[label=(\alph*)]
\item Scaling gives \(V_n(a)=a^n V_n(1)\).
\item Slice by \(x_1\) and use the \((n-1)\)-dimensional volume of the cross-section \(\{(x_2,\ldots,x_n): |x_2|+\cdots+|x_n|\le 1-|x_1|\}\). Induction yields
\(V_n(1)=2\,V_{n-1}(1)\int_0^1 (1-x)^{n-1}dx=2\,V_{n-1}(1)/n\), hence \(V_n(1)=2^n/n!\).
\end{enumerate}\qed


\begin{problembox}[15.14: Volume of Special \( n \)-Dimensional Set]
\begin{problemstatement}
If \( a > 0 \) and \( n \geq 2 \), let \( S_n(a) \) denote the following set in \( \mathbb{R}^n \):
\[
S_n(a) = \{(x_1, \ldots, x_n): |x_i| + |x_n| \leq a \quad \text{for each } i = 1, \ldots, n - 1\}.
\]
Let \( V_n(a) \) denote the \( n \)-measure of \( S_n(a) \). Use a method suggested by Exercise 15.13 to prove that \( V_n(a) = 2^n a^n / n \).
\end{problemstatement}
\end{problembox}

\noindent\textbf{Strategy:} Fix the last coordinate \(x_n\) and observe that the cross-section in the first \(n-1\) coordinates forms an \((n-1)\)-dimensional diamond. Use the result from Exercise 15.13 for the volume of this cross-section, then integrate over \(x_n\).

\bigskip\noindent\textbf{Solution:}
Fix \(x_n\in[-a,a]\). The cross-section in the first \(n-1\) coordinates is an \((n-1)\)-dimensional diamond of radius \(a-|x_n|\) with volume \(V_{n-1}(a-|x_n|)=2^{n-1}(a-|x_n|)^{n-1}/(n-1)!\). Integrating in \(x_n\) gives
\[
V_n(a)=\int_{-a}^{a} \frac{2^{n-1}}{(n-1)!} (a-|t|)^{n-1} dt = \frac{2^n a^n}{n}.
\]\qed


\begin{problembox}[15.15: Integral over First Quadrant of \( n \)-Ball]
\begin{problemstatement}
Let \( Q_n(a) \) denote the ``first quadrant'' of the \( n \)-ball \( B(0:a) \) given by
$Q_n(a) = \{(x_1, \ldots, x_n): \|x\| \leq a$ and $ 0 \leq x_i \leq a $ for each $ i = 1, 2, \ldots, n.$
Let \( f(x) = x_1 \cdots x_n \) and prove that
\[
\int_{Q_n(a)} f(x) \, dx = \frac{a^{2n}}{2^n n!}.
\]
\end{problemstatement}
\end{problembox}

\noindent\textbf{Strategy:} Use the change of variables \(y_i = x_i^2\) to transform the region into a simplex. The Jacobian simplifies the integrand, and the volume of the simplex can be computed using the formula for the volume of an \(n\)-dimensional simplex.

\bigskip\noindent\textbf{Solution:}
Let \(y_i=x_i^2\) for \(i=1,\ldots,n\). On the first quadrant, \(x_i\ge 0\), the region \(Q_n(a)\) maps to the simplex \(\{y\ge 0: y_1+\cdots+y_n\le a^2\}\). The Jacobian gives \(dx=\frac{1}{2^n}(y_1\cdots y_n)^{-1/2}dy\) and \(x_1\cdots x_n=\sqrt{y_1\cdots y_n}\). Therefore the integrand times \(dx\) is \(\frac{1}{2^n} dy\), and
\[
\int_{Q_n(a)} x_1\cdots x_n\,dx = \frac{1}{2^n}\,\mathrm{Vol}\,\{y\ge 0: y_1+\cdots+y_n\le a^2\} = \frac{a^{2n}}{2^n n!}.
\]

\begin{techniquessection}[Solving and Proving Techniques]

\subsection*{Essential Definitions and Theorems}

\begin{definition}[Fubini's Theorem Strategy]
Fubini's theorem allows us to compute multiple integrals as iterated integrals when the function is absolutely integrable. The key insight is that the order of integration can be interchanged without affecting the result.
\end{definition}

\noindent\textbf{Importance:} Fubini's theorem is the fundamental tool for computing multiple integrals. It reduces complex multidimensional problems to simpler one-dimensional integrals, making many calculations feasible. The ability to interchange integration order is crucial for choosing the most convenient order for computation.



\begin{definition}[Change of Variables Strategy]
The change of variables technique transforms integrals over complex regions into integrals over simpler regions by using coordinate transformations. The Jacobian determinant accounts for how the transformation changes volumes locally.
\end{definition}

\noindent\textbf{Importance:} Change of variables is essential for computing integrals over regions that are naturally described in different coordinate systems. It allows us to exploit symmetry and choose coordinate systems that simplify the integrand or the region of integration.



\begin{theorem}[Slicing Method]
For computing volumes of complex regions, slice the region by fixing one coordinate and compute the volume as an integral of cross-sectional areas or volumes.
\end{theorem}

\noindent\textbf{Importance:} The slicing method is a powerful geometric technique that reduces volume calculations to lower-dimensional problems. It's particularly useful for regions with symmetry or special structure, and provides an intuitive way to understand complex geometric objects.



\begin{theorem}[Scaling and Symmetry]
Use scaling properties to relate volumes of similar regions and exploit symmetry to simplify calculations by reducing the region of integration.
\end{theorem}

\noindent\textbf{Importance:} Scaling and symmetry are fundamental principles that greatly simplify volume calculations. Scaling allows us to compute volumes for different sizes from a single calculation, while symmetry reduces the computational effort by allowing us to focus on fundamental regions.



\begin{theorem}[Special Functions and Recursion]
Use special functions like the Gamma function and recursion formulas to compute volumes in higher dimensions and establish relationships between different dimensional volumes.
\end{theorem}

\noindent\textbf{Importance:} Special functions and recursion provide systematic ways to compute volumes in higher dimensions. The Gamma function naturally appears in volume formulas, while recursion allows us to build up solutions from lower-dimensional cases.



\subsection*{Working with Lebesgue Integrals}
\begin{itemize}
\item Use the fact that Lebesgue integrals extend Riemann integrals
\item Apply the fact that Lebesgue integrals are linear and monotone
\item Use the fact that Lebesgue integrals can handle unbounded functions and infinite regions
\item Apply the fact that Lebesgue integrals are invariant under changes of variables
\item Use the fact that Lebesgue integrals can be computed as limits of simple functions
\end{itemize}

\subsection*{Applying Change of Variables}
\begin{itemize}
\item Use the fact that the Jacobian determinant gives the scaling factor for volume
\item Apply the fact that linear transformations scale volumes by the determinant
\item Use the fact that polar and spherical coordinates have known Jacobians
\item Apply the fact that changes of variables preserve measurability
\item Use the fact that the Jacobian can be computed from the derivative matrix
\end{itemize}

\subsection*{Working with Slicing}
\begin{itemize}
\item Use the fact that the measure of a set equals the integral of its cross-sectional measures
\item Apply the fact that slicing can be done in any coordinate direction
\item Use the fact that slicing preserves measurability
\item Apply the fact that slicing can be used to compute volumes of complex regions
\item Use the fact that slicing can be used to prove geometric formulas
\end{itemize}

\subsection*{Computing Volumes in Higher Dimensions}
\begin{itemize}
\item Use the fact that $n$-ball volumes can be computed using spherical coordinates
\item Apply the fact that volumes scale by the $n$th power of the scaling factor
\item Use the fact that volumes can be computed by slicing into lower-dimensional regions
\item Apply the fact that symmetry can be used to simplify volume calculations
\item Use the fact that volumes can be computed using recursion formulas
\end{itemize}

\subsection*{Working with Special Functions}
\begin{itemize}
\item Use the Gamma function: $\Gamma(n+1) = n!$ for positive integers
\item Apply the fact that $\Gamma(1/2) = \sqrt{\pi}$
\item Use the fact that Beta functions can be expressed in terms of Gamma functions
\item Apply the fact that special functions can be used to evaluate difficult integrals
\item Use the fact that recursion formulas can be used to compute special function values
\end{itemize}
\end{techniquessection}