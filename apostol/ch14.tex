\chapter{Multiple Riemann Integrals}
\section{Multiple Integrals}

\subsection*{Essential Definitions and Theorems}

\begin{definition}[Riemann Integral on Rectangles]
For a bounded function $f$ on a rectangle $Q = [a_1, b_1] \times \cdots \times [a_n, b_n]$, the Riemann integral is defined as the common value of the upper and lower Darboux sums when they exist:
\[\int_Q f = \inf\{U(f, P) : P \text{ is a partition}\} = \sup\{L(f, P) : P \text{ is a partition}\}\]
\end{definition}

\noindent\textbf{Importance:} This definition extends the concept of Riemann integration to higher dimensions. The integrability criterion based on vanishing total oscillation provides a practical way to determine when functions are integrable. This is the foundation for multiple integration theory.



\begin{definition}[Upper and Lower Darboux Sums]
For a partition $P$ of a rectangle $Q$, the upper and lower Darboux sums are:
\[U(f, P) = \sum_{R \in P} M_R(f) \cdot \text{vol}(R), \quad L(f, P) = \sum_{R \in P} m_R(f) \cdot \text{vol}(R)\]
where $M_R(f) = \sup\{f(x) : x \in R\}$ and $m_R(f) = \inf\{f(x) : x \in R\}$.
\end{definition}

\noindent\textbf{Importance:} These sums provide the fundamental tools for defining and computing multiple integrals. They give upper and lower bounds for the integral and are essential for proving integrability criteria and convergence results.



\begin{theorem}[Product Formula for Separable Integrands]
If $f \in R[a, b]$ and $g \in R[c, d]$, then the product function $h(x, y) = f(x)g(y)$ is Riemann integrable on $[a, b] \times [c, d]$ and:
\[\int_{[a,b] \times [c,d]} f(x)g(y) \, d(x, y) = \left(\int_a^b f(x) \, dx\right) \left(\int_c^d g(y) \, dy\right)\]
\end{theorem}

\noindent\textbf{Importance:} This theorem allows us to factor multiple integrals when the integrand separates into a product of functions of different variables. It's essential for computing many integrals and is the foundation for Fubini's theorem and iterated integration.



\begin{theorem}[Integrability of Monotone Functions]
If a bounded function $f$ on a rectangle $Q$ is monotone in each variable separately, then $f$ is Riemann integrable on $Q$.
\end{theorem}

\noindent\textbf{Importance:} This theorem provides a large class of functions that are guaranteed to be integrable. Monotone functions are common in applications, and this result ensures that many practical problems can be solved using Riemann integration.



\begin{theorem}[Fubini's Theorem for Riemann Integrals]
If $f$ is continuous (or piecewise continuous) on a rectangle $Q = [a, b] \times [c, d]$, then:
\[\int_Q f(x, y) \, d(x, y) = \int_a^b \left(\int_c^d f(x, y) \, dy\right) dx = \int_c^d \left(\int_a^b f(x, y) \, dx\right) dy\]
\end{theorem}

\noindent\textbf{Importance:} This theorem allows us to compute multiple integrals by iterated integration, which is often much easier than direct computation. It's the fundamental tool for evaluating double and triple integrals and is essential for applications in physics, engineering, and probability.





\begin{problembox}[14.1: Product of Riemann Integrable Functions]
\begin{problemstatement}
If \( f_1 \in R \) on \([a_1, b_1], \ldots, f_n \in R \) on \([a_n, b_n]\), prove that
\begin{align*}
 &\int_{S} f_1(x_1) \cdots f_n(x_n) \, d(x_1, \ldots, x_n) \\
 =& \left( \int_{a_1}^{b_1} f_1(x_1) \, dx_1 \right) \cdots \left( \int_{a_n}^{b_n} f_n(x_n) \, dx_n \right), 
\end{align*}
 where \( S = [a_1, b_1] \times \cdots \times [a_n, b_n] \).
\end{problemstatement}
\end{problembox}

\noindent\textbf{Strategy:} Use the product formula for separable integrands. For $n=2$, show that the product function $g(x,y)=f_1(x)f_2(y)$ is Riemann integrable by bounding the oscillation on product partitions. Then apply induction for $n>2$ by grouping variables two at a time.

\bigskip\noindent\textbf{Solution:}
For $n=2$, write $g(x,y)=f_1(x)f_2(y)$. Since $f_1\in R[a_1,b_1]$ and $f_2\in R[a_2,b_2]$, for any $\varepsilon>0$ there are partitions $\mathcal P_1,\mathcal P_2$ such that $U(f_i,\mathcal P_i)-L(f_i,\mathcal P_i)<\varepsilon$ $(i=1,2)$. For the product partition $\mathcal P=\mathcal P_1\times\mathcal P_2$ on $S$, the oscillation of $g$ on each rectangle factors, and one obtains
\begin{align*}
 U(g,\mathcal P)-L(g,\mathcal P) \le & (U(f_1,\mathcal P_1)-L(f_1,\mathcal P_1))\,\int f_2 + (U(f_2,\mathcal P_2) \\
 &-L(f_2,\mathcal P_2))\,\int f_1 + O(\varepsilon^2), 
\end{align*}
which can be made arbitrarily small. Hence $g\in R$ and
\begin{align*}
 \iint_S f_1(x)f_2(y)\,d(x,y) = \int_{a_1}^{b_1} f_1(x)\,dx\,\int_{a_2}^{b_2} f_2(y)\,dy. 
\end{align*}
The case $n>2$ follows by induction by grouping variables two at a time and applying the $n=2$ case repeatedly.\qed


\begin{problembox}[14.2: Riemann Integrability of Monotone Functions]
\begin{problemstatement}
Let \( f \) be defined and bounded on a compact rectangle \( Q = [a, b] \times [c, d] \) in \( \mathbb{R}^2 \). Assume that for each fixed \( y \) in \([c, d]\), \( f(x, y) \) is an increasing function of \( x \), and that for each fixed \( x \) in \([a, b]\), \( f(x, y) \) is an increasing function of \( y \). Prove that \( f \in R \) on \( Q \).
\end{problemstatement}
\end{problembox}

\noindent\textbf{Strategy:} Use the integrability criterion via vanishing total oscillation. Partition the rectangle into subintervals of small mesh and bound the oscillation on each subrectangle using the monotonicity assumption. Show that the total oscillation can be made arbitrarily small.

\bigskip\noindent\textbf{Solution:}
Partition $[a,b]$ and $[c,d]$ into subintervals of mesh smaller than $\delta>0$. On each rectangle $R=I\times J$, monotonicity in each variable gives
\[ \operatorname{osc}(f;R) \le \max_{x\in I}\!f(x,\sup J)-\min_{x\in I}\!f(x,\inf J) \le f(\sup I,\sup J)-f(\inf I,\inf J). \]
Summing over the grid yields
\[ U(f)-L(f) \le \sum_{i,j} \big(f(x_{i+1},y_{j+1})-f(x_i,y_j)\big) \le (\text{Var}_x f)\,\delta + (\text{Var}_y f)\,\delta, \]
which can be made $<\varepsilon$ by choosing $\delta$ small. Thus $U(f)=L(f)$ and $f\in R$ on $Q$.\qed


\begin{problembox}[14.3: Evaluation of Double Integrals]
\begin{problemstatement}
Evaluate each of the following double integrals.
\begin{enumerate}[label=(\alph*)]
    \item \[ \iint_{Q} \sin^2 x \, \sin^2 y \, dx \, dy, \quad \text{where } Q = [0, \pi] \times [0, \pi]. \]
    \item \[ \iint_{Q} |\cos (x + y)| \, dx \, dy, \quad \text{where } Q = [0, \pi] \times [0, \pi]. \]
    \item $$ \iint_{Q} [x + y] \, dx \, dy,$$  where $Q = [0, 2] \times [0, 2]$, and $[t]$  is the greatest integer  $\leq t$.
\end{enumerate}
\end{problemstatement}
\end{problembox}

\noindent\textbf{Strategy:} For (a), use the product formula since the integrand factors. For (b), use a change of variables $t=x+y$ and compute the fiber length function. For (c), use the floor function to break the integral into regions where $[x+y]$ is constant.

\bigskip\noindent\textbf{Solution:}
\begin{enumerate}[label=(\alph*)]
    \item The integrand factors, so
    \[ \iint_Q \sin^2 x\,\sin^2 y\,dx\,dy = \left(\int_0^{\pi} \sin^2 x\,dx\right)^2 = \left(\tfrac{\pi}{2}\right)^2 = \tfrac{\pi^2}{4}. \]
    \item Let $t=x+y$. For $t\in[0,\pi]$ the fiber length is $t$; for $t\in[\pi,2\pi]$ it is $2\pi-t$. Hence
    \[ \iint_Q |\cos(x+y)|\,dx\,dy = \int_0^{2\pi} |\cos t|\,m(t)\,dt = 2\int_0^{\pi} t\,|\cos t|\,dt = 2\pi. \]
    \item With $m(t)$ the fiber length in $[0,2]^2$, $m(t)=t$ on $[0,2]$ and $m(t)=4-t$ on $[2,4]$. Thus
    \[ \iint_Q [x+y]\,dx\,dy = \sum_{k=0}^3 k\int_k^{k+1} m(t)\,dt = 0\cdot\tfrac{1}{2} + 1\cdot\tfrac{3}{2} + 2\cdot\tfrac{3}{2} + 3\cdot\tfrac{1}{2} = 6. \]
\end{enumerate}\qed


\begin{problembox}[14.4: Integrals over Unit Square]
\begin{problemstatement}
Let \( Q = [0, 1] \times [0, 1] \) and calculate \( \int_{Q} f(x, y) \, dx \, dy \) in each case.
\begin{enumerate}[label=(\alph*)]
    \item \( f(x, y) = 1 - x - y \) if \( x + y \leq 1, \quad f(x, y) = 0 \) otherwise.
    \item \( f(x, y) = x^2 + y^2 \) if \( x^2 + y^2 \leq 1, \quad f(x, y) = 0 \) otherwise.
    \item \( f(x, y) = x + y \) if \( x^2 \leq y \leq 2x^2, \quad f(x, y) = 0 \) otherwise.
\end{enumerate}
\end{problemstatement}
\end{problembox}

\noindent\textbf{Strategy:} For (a), integrate over the triangular region where $x+y\leq 1$. For (b), use polar coordinates since the region is a quarter-circle. For (c), find the intersection points of the curves $y=x^2$ and $y=2x^2$ with $y=1$ to determine the integration limits.

\bigskip\noindent\textbf{Solution:}
\begin{enumerate}[label=(\alph*)]
    \item Over $\{x+y\le 1\}$,
    \[ \int_0^1 \!\int_0^{1-x} (1-x-y)\,dy\,dx = \int_0^1 \tfrac{(1-x)^2}{2}\,dx = \tfrac{1}{6}. \]
    \item Quarter-disk of radius 1: in polar coordinates,
    \[ \int_0^{\pi/2}\!\int_0^1 (x^2+y^2)\,r\,dr\,d\theta = \int_0^{\pi/2}\!\int_0^1 r^3\,dr\,d\theta = \tfrac{\pi}{8}. \]
    \item Decompose by $x$:
    \[ \int_0^{1/\sqrt{2}} \!\int_{x^2}^{2x^2} (x+y)\,dy\,dx\; +\; \int_{1/\sqrt{2}}^{1} \!\int_{x^2}^{1} (x+y)\,dy\,dx. \]
    The first term equals $\int_0^{1/\sqrt{2}} \big(x^3+\tfrac{3}{2}x^4\big)dx= \tfrac{1}{16}+\tfrac{3}{40\sqrt{2}}$. The second term equals
    \[ \int_{1/\sqrt{2}}^{1} \Big(x(1-x^2)+\tfrac{1-x^4}{2}\Big)dx = \left[\tfrac{x^2}{2}-\tfrac{x^4}{4}+\tfrac{x}{2}-\tfrac{x^5}{10}\right]_{1/\sqrt{2}}^{1} = \tfrac{37}{80}-\tfrac{19}{40\sqrt{2}}. \]
    Summing gives $\tfrac{21}{40}-\tfrac{2}{5\sqrt{2}}$.
\end{enumerate}\qed


\begin{problembox}[14.5: Mixed Partial Integrals]
\begin{problemstatement}
Define \( f \) on the square \( Q = [0, 1] \times [0, 1] \) as follows:
\[ f(x, y) = 
\begin{cases} 
1 & \text{if } x \text{ is rational}, \\
2y & \text{if } x \text{ is irrational}. 
\end{cases} \]
\begin{enumerate}[label=(\alph*)]
    \item Prove that \( \int_{0}^{t} f(x, y) \, dy \) exists for \( 0 \leq t \leq 1 \) and that
    \[ \underline\int_{0}^{1} \left[ \int_{0}^{t} f(x, y) \, dy \right] \, dx = t^2, \]
    and \[ \overline\int_{0}^{1} \left[ \int_{0}^{t} f(x, y) \, dy \right] \, dx = t. \]
    This shows that \( \int_{0}^{1} \left[ \int_{0}^{1} f(x, y) \, dy \right] \, dx \) exists and equals 1.
    
    \item Prove that \( \int_{0}^{1} \left[ \overline\int_{0}^{1} f(x, y) \, dx \right] \, dy \) exists and find its value.
    \item Prove that the double integral \( \int_{Q} f(x, y) \, d(x, y) \) does not exist.
\end{enumerate}
\end{problemstatement}
\end{problembox}

\noindent\textbf{Strategy:} For (a), compute the inner integral for fixed $x$ and show it takes two values on dense sets. For (b), compute the upper integral in $x$ for fixed $y$. For (c), show that every rectangle has oscillation 1, so lower and upper sums cannot agree.

\bigskip\noindent\textbf{Solution:}
\begin{enumerate}[label=(\alph*)]
\item For fixed $x$ and $t\in[0,1]$, the function $y\mapsto f(x,y)$ is Riemann integrable on $[0,t]$ and
\[\int_0^t f(x,y)\,dy = \begin{cases} t, & x\in\mathbb Q, \\[4pt] t^2, & x\notin\mathbb Q.\end{cases}\]
As a function of $x$, this takes the two values $t$ and $t^2$ on dense sets. Hence on every subinterval of $[0,1]$ the supremum is $\max\{t,t^2\}=t$ and the infimum is $\min\{t,t^2\}=t^2$. Therefore
\[ \underline\int_0^1\!\Big[\int_0^t f(x,y)\,dy\Big]dx = t^2,\qquad \overline\int_0^1\!\Big[\int_0^t f(x,y)\,dy\Big]dx = t. \]
In particular, for $t=1$ we have $\int_0^1\!\big[\int_0^1 f(x,y)\,dy\big]dx=1$ since both cases give the value $1$.

\item For fixed $y\in[0,1]$, as a function of $x$ the values $1$ and $2y$ occur on dense sets, so on every subinterval the supremum is $\max\{1,2y\}$ and the infimum is $\min\{1,2y\}$. Hence the upper integral in $x$ exists and equals
\[ \overline\int_0^1 f(x,y)\,dx = \max\{1,2y\} = \begin{cases} 1, & 0\le y\le \tfrac12, \\[4pt] 2y, & \tfrac12< y\le 1. \end{cases} \]
Thus
\[ \int_0^1\!\Big[\overline\int_0^1 f(x,y)\,dx\Big]dy = \int_0^{1/2}\!1\,dy + \int_{1/2}^1\!2y\,dy = \tfrac12 + (1-\tfrac14) = \tfrac{5}{4}. \]

\item Let $R=I\times J$ with $J=[\alpha,\beta]$. Because rationals and irrationals in $x$ are dense, we have
\[ \sup_R f = \max\{1,2\beta\},\qquad \inf_R f=\min\{1,2\alpha\}. \]
Consequently, for any partition $\mathcal P$ of $Q$,
\[ L(f,\mathcal P) \le \int_0^1 \min\{1,2y\}\,dy = \tfrac{3}{4},\qquad U(f,\mathcal P) \ge \int_0^1 \max\{1,2y\}\,dy = \tfrac{5}{4}. \]
Hence $\underline{\iint_Q} f \le \tfrac34 < \tfrac54 \le \overline{\iint_Q} f$, so the double Riemann integral $\int_Q f$ does not exist.
\end{enumerate}\qed


\begin{problembox}[14.6: Discontinuous Integrand]
\begin{problemstatement}
Define \( f \) on the square \( Q = [0, 1] \times [0, 1] \) as follows:
\[f(x, y) = 
\begin{cases} 
0 & \text{if at least one of } x, y \text{ is irrational}, \\ 
1/n & \text{if } y \text{ is rational and } x = m/n,
\end{cases}\]
where \( m \) and \( n \) are relatively prime integers, \( n > 0 \). Prove that
\[\int_{0}^{1} f(x, y) \, dx = \int_{0}^{1} \left[ \int_{0}^{1} f(x, y) \, dx \right] \, dy = \int_{Q} f(x, y) \, d(x, y) = 0\]
but that \( \int_{0}^{1} f(x, y) \, dy \) does not exist for rational \( x \).
\end{problemstatement}
\end{problembox}

\noindent\textbf{Strategy:} Show that for fixed $y$, the function $f(\cdot,y)$ is zero except on a countable set, making the integral zero. For the double integral, show that the set where $f\neq 0$ has content zero. For rational $x$, show that $f(x,\cdot)$ is not Riemann integrable.

\bigskip\noindent\textbf{Solution:}
For fixed $y$, $f(\cdot,y)$ is zero except possibly on the countable set $\{m/n\}$ when $y$ is rational. Given $\varepsilon>0$, choose a partition of $[0,1]$ so that the total length of intervals covering those rationals is $<\varepsilon$; the contribution to upper sums is then $<\varepsilon\cdot\sup f\le \varepsilon$, hence $\int_0^1 f(x,y)\,dx=0$. Therefore $\int_0^1\!\Big[\int_0^1 f(x,y)\,dx\Big]dy=0$.
Similarly, on any rectangle in $Q$ the set where $f\ne 0$ is a countable subset with total contribution to upper sums made arbitrarily small, so $\int_Q f=0$. However, for rational $x=m/n$, the function $y\mapsto f(m/n,y)$ equals $\tfrac{1}{n}$ on rationals and $0$ on irrationals, which is not Riemann integrable, so $\int_0^1 f(x,y)\,dy$ does not exist for rational $x$.\qed


\begin{problembox}[14.7: Dense Set with Finite Cross-Sections]
\begin{problemstatement}
If \( p_k \) denotes the \( k \)th prime number, let
\[S(p_k) = \left\{ \begin{pmatrix}
n & m \\
p_k & p_k
\end{pmatrix} : n = 1, 2, \ldots, p_k - 1, \quad m = 1, 2, \ldots, p_k - 1 \right\},\]
let \( S = \bigcup_{k=1}^{\infty} S(p_k) \), and let \( Q = [0, 1] \times [0, 1] \).

\begin{enumerate}[label=(\alph*)]
    \item Prove that \( S \) is dense in \( Q \) (that is, the closure of \( S \) contains \( Q \)) but that any line parallel to the coordinate axes contains at most a finite subset of \( S \).
    
    \item Define \( f \) on \( Q \) as follows:
    \[f(x, y) = 0 \quad \text{if } (x, y) \in S, \quad f(x, y) = 1 \quad \text{if } (x, y) \in Q - S.\]
    Prove that \( \int_{0}^{1} \left[ \int_{0}^{1} f(x, y) \, dy \right] \, dx = \int_{0}^{1} \left[ \int_{0}^{1} f(x, y) \, dx \right] \, dy = 1 \), but that the double integral \( \int_{Q} f(x, y) \, d(x, y) \) does not exist.
\end{enumerate}
\end{problemstatement}
\end{problembox}

\noindent\textbf{Strategy:} For (a), show that the grid points become arbitrarily dense as primes increase, but each line meets only finitely many points. For (b), show that vertical sections have content zero, making iterated integrals exist, but every rectangle has oscillation 1.

\bigskip\noindent\textbf{Solution:}
\begin{enumerate}[label=(\alph*)]
    \item For any rectangle in $Q$ and any large prime $p$, the grid $\{(n/p,m/p):1\le n,m\le p-1\}$ lies $1/p$-dense, hence $S$ is dense. A vertical line $x=x_0$ meets $S$ only when $x_0=n/p$ with $p$ prime; for a reduced rational $x_0=a/b$, this forces $b$ prime and yields at most $p-1$ points, otherwise none. Similarly for horizontal lines; hence each axis-parallel line meets $S$ in a finite set.
    \item For fixed $x$, the vertical section $\{y:(x,y)\in S\}$ is finite, so it has Jordan content zero and $\int_0^1 f(x,y)\,dy=1$. Thus $\int_0^1\![\int_0^1 f(x,y)dy]dx=1$. The same holds with $x$ and $y$ interchanged. However, every rectangle contains points of $S$ and of $Q\setminus S$, so the oscillation of $f$ on every subrectangle is $1$; lower sums are $0$ and upper sums are $1$, hence $\int_Q f$ does not exist.
\end{enumerate}\qed
\section{Jordan Content}

\subsection*{Essential Definitions and Theorems}

\begin{definition}[Jordan Outer and Inner Content]
For a bounded set $S \subset \mathbb{R}^n$, the Jordan outer content is:
\[c^*(S) = \inf\{\sum_{k=1}^m \text{vol}(Q_k) : S \subset \bigcup_{k=1}^m Q_k, Q_k \text{ are rectangles}\}\]
The Jordan inner content is:
\[c_*(S) = \sup\{\sum_{k=1}^m \text{vol}(Q_k) : \bigcup_{k=1}^m Q_k \subset S, Q_k \text{ are disjoint rectangles}\}\]
\end{definition}

\noindent\textbf{Importance:} Jordan content provides a way to measure the "size" of bounded sets in $\mathbb{R}^n$ and is a precursor to Lebesgue measure.

\begin{definition}[Jordan-Measurable Sets]
A bounded set $S \subset \mathbb{R}^n$ is Jordan-measurable if $c^*(S) = c_*(S)$. The common value is called the Jordan content of $S$, denoted $c(S)$.
\end{definition}

\noindent\textbf{Importance:} Jordan-measurable sets are those for which we can meaningfully assign a "volume" or "content" and form the natural domain for Riemann integration.

\begin{theorem}[Jordan-Measurability Criterion]
A bounded set $S \subset \mathbb{R}^n$ is Jordan-measurable if and only if its boundary $\partial S$ has Jordan content zero.
\end{theorem}

\noindent\textbf{Importance:} This criterion provides a practical way to determine whether a set is Jordan-measurable. It shows that the "thickness" of the boundary is what determines measurability. This is essential for understanding which regions can be used for integration.



\begin{theorem}[Sets of Content Zero]
The following sets have Jordan content zero:
\begin{enumerate}[label=(\alph*)]
\item Countable sets in $\mathbb{R}^n$
\item Graphs of continuous functions $f: [a, b] \to \mathbb{R}$
\item Rectifiable curves in $\mathbb{R}^n$
\item Images of sets of content zero under Lipschitz functions
\end{enumerate}
\end{theorem}

\noindent\textbf{Importance:} This theorem identifies important classes of sets that have zero content. These results are essential for integration theory, as they show that "thin" sets don't contribute to integrals. The fact that graphs of continuous functions have zero content is particularly important.



\begin{theorem}[Cavalieri's Principle for Jordan Content]
If $S$ is a Jordan-measurable region in $\mathbb{R}^n$ and $f$ is continuous on $S$, then the ordinate set of $f$ over $S$ has $(n+1)$-dimensional Jordan content equal to:
\[\int_S f(x_1, \ldots, x_n) \, d(x_1, \ldots, x_n)\]
\end{theorem}

\noindent\textbf{Importance:} This principle connects integration with geometric content, showing that the integral of a function over a region equals the volume of the region "under the graph" of the function. This is the foundation for volume calculations and many applications in physics and engineering.





\begin{problembox}[14.8: Jordan Content of Finite Accumulation Points]
\begin{problemstatement}
Let \( S \) be a bounded set in \( \mathbb{R}^n \) having at most a finite number of accumulation points. Prove that \( c(S) = 0 \).
\end{problemstatement}
\end{problembox}

\noindent\textbf{Strategy:} Cover the accumulation points with small cubes and the remaining isolated points with disjoint cubes. Show that the total content can be made arbitrarily small.

\bigskip\noindent\textbf{Solution:}
Let $\{a_1,\dots,a_m\}$ be the accumulation points (possibly empty). Cover each $a_j$ by a cube of content $<\varepsilon/(2m)$. The remaining points of $S$ are isolated; choose disjoint cubes around each, with total content $<\varepsilon/2$ (possible since only finitely many lie outside any given compact exhaustion). Thus $S$ is covered by cubes of total content $<\varepsilon$. As $\varepsilon>0$ is arbitrary, $c(S)=0$.\qed


\begin{problembox}[14.9: Graph of Continuous Function has Zero Content]
\begin{problemstatement}
Let \( f \) be a continuous real-valued function defined on \([a, b]\). Let \( S \) denote the graph of \( f \), that is, \( S = \{(x, y) : y = f(x), a \leq x \leq b\} \). Prove that \( S \) has two-dimensional Jordan content zero.
\end{problemstatement}
\end{problembox}

\noindent\textbf{Strategy:} Use uniform continuity to partition the domain into small intervals and cover the graph over each interval with thin rectangles. Show the total area can be made arbitrarily small.

\bigskip\noindent\textbf{Solution:}
By uniform continuity of $f$ on $[a,b]$, for $\varepsilon>0$ choose $\delta$ so that $|x-x'|<\delta$ implies $|f(x)-f(x')|<\varepsilon/(b-a)$. Partition $[a,b]$ into subintervals of length $<\delta$ and cover the graph over each subinterval by a rectangle of width $\Delta x$ and height $<\varepsilon/(b-a)$. The total area is $<\varepsilon$. Hence the graph has content zero.\qed


\begin{problembox}[14.10: Rectifiable Curve has Zero Content]
\begin{problemstatement}
Let \( \Gamma \) be a rectifiable curve in \( \mathbb{R}^n \). Prove that \( \Gamma \) has \( n \)-dimensional Jordan content zero.
\end{problemstatement}
\end{problembox}

\noindent\textbf{Strategy:} Approximate the curve by a polygonal path and cover each segment with thin tubes. Show that the total content can be made arbitrarily small by choosing thin enough tubes.

\bigskip\noindent\textbf{Solution:}
Approximate $\Gamma$ by a polygonal path of length within $\varepsilon$ of its total length $L$. Cover each segment by a tube of thickness $\eta>0$; the $n$-dimensional content of the tube is bounded by $C_n\,L\,\eta$, which can be made arbitrarily small by choosing $\eta$. Therefore $c(\Gamma)=0$.\qed


\begin{problembox}[14.11: Ordinate Set Content]
\begin{problemstatement}
Let \( f \) be a nonnegative function defined on a set \( S \) in \( \mathbb{R}^n \). The ordinate set of \( f \) over \( S \) is defined to be the following subset of \( \mathbb{R}^{n+1} \):
\[\{(x_1, \ldots, x_n, x_{n+1}) : (x_1, \ldots, x_n) \in S, \quad 0 \leq x_{n+1} \leq f(x_1, \ldots, x_n)\}.\]
If \( S \) is a Jordan-measurable region in \( \mathbb{R}^n \) and if \( f \) is continuous on \( S \), prove that the ordinate set of \( f \) over \( S \) has \( (n + 1) \)-dimensional Jordan content whose value is
\[\int_{S} f(x_1, \ldots, x_n) \, d(x_1, \ldots, x_n).\]
Interpret this problem geometrically when \( n = 1 \) and \( n = 2 \).
\end{problemstatement}
\end{problembox}

\noindent\textbf{Strategy:} Use Cavalieri's principle: the content equals the integral of the section lengths. For each $x\in S$, the vertical section has length $f(x)$.

\bigskip\noindent\textbf{Solution:}
Vertical sections at $x\in S$ are intervals of length $f(x)$; outside $S$ they are empty. By continuity of $f$ and Jordan-measurability of $S$, the ordinate set is Jordan-measurable and its content equals the integral of section lengths:
\[ c_{n+1}(\text{ord}(f,S)) = \int_S f(x)\,dx. \]
Geometrically: for $n=1$, the area under the curve $y=f(x)$ over $S\subset\mathbb R$; for $n=2$, the volume under the surface $z=f(x,y)$ over a planar region $S$.\qed
\section{Advanced Topics}

\subsection*{Essential Definitions and Theorems}

\begin{definition}[Almost Everywhere Properties]
A property holds almost everywhere (a.e.) on a set $S$ if it holds on $S$ except possibly on a set of Jordan content zero. For Riemann integrable functions, this concept is essential for understanding when functions are equal or have certain properties.
\end{definition}

\noindent\textbf{Importance:} The concept of "almost everywhere" allows us to ignore sets of zero content when working with integrals. This is crucial for many results in integration theory, as it provides a way to handle functions that may differ on negligible sets without affecting the integral.



\begin{theorem}[Mean Value Theorem for Multiple Integrals]
If $f$ is continuous on a Jordan-measurable region $S \subset \mathbb{R}^n$, then there exists an interior point $x_0 \in S$ such that:
\[\int_S f(x) \, dx = f(x_0) \cdot c(S)\]
\end{theorem}

\noindent\textbf{Importance:} This theorem extends the fundamental mean value theorem to multiple integrals. It shows that the integral of a continuous function over a region equals the value of the function at some point times the content of the region. This is essential for understanding average values and for many applications in physics and engineering.



\begin{theorem}[Equality of Mixed Partial Derivatives]
If $f$ is continuous on a rectangle $Q = [a, b] \times [c, d]$, then for the iterated integral:
\[F(x_1, x_2) = \int_a^{x_1} \left(\int_c^{x_2} f(x, y) \, dy\right) dx\]
we have:
\[D_{1,2} F(x_1, x_2) = D_{2,1} F(x_1, x_2) = f(x_1, x_2)\]
\end{theorem}

\noindent\textbf{Importance:} This theorem shows that the order of differentiation doesn't matter for iterated integrals of continuous functions. It's a key result connecting integration and differentiation in multiple variables and is essential for many applications in differential equations and analysis.



\begin{theorem}[Fundamental Theorem of Calculus for Multiple Variables]
If $f$ is continuous on a rectangle $Q$, then the iterated integral:
\[F(x_1, x_2) = \int_a^{x_1} \int_c^{x_2} f(x, y) \, dy \, dx\]
satisfies:
\[D_1 F(x_1, x_2) = \int_c^{x_2} f(x_1, y) \, dy, \quad D_2 F(x_1, x_2) = \int_a^{x_1} f(x, x_2) \, dx\]
\end{theorem}

\noindent\textbf{Importance:} This theorem extends the fundamental theorem of calculus to multiple variables. It shows how to differentiate iterated integrals and is essential for understanding the relationship between integration and differentiation in higher dimensions.





\begin{problembox}[14.12: Zero Integral Implies Zero Function]
\begin{problemstatement}
Assume that \( f \in R \) on \( S \) and suppose \( \int_S f(x) \, dx = 0 \). (\( S \) is a subset of \( \mathbb{R}^n \)). Let \( A = \{ x : x \in S, f(x) < 0 \} \) and assume that \( c(A) = 0 \). Prove that there exists a set \( B \) of measure zero such that \( f(x) = 0 \) for each \( x \) in \( S - B \).
\end{problemstatement}
\end{problembox}

\noindent\textbf{Strategy:} Use contradiction: if the set where $f>\varepsilon$ has positive content for some $\varepsilon>0$, then the integral would be positive. Show that the set where $f>0$ has content zero.

\bigskip\noindent\textbf{Solution:}
Let $E_\varepsilon=\{x\in S:f(x)>\varepsilon\}$. If $c(E_\varepsilon)>0$ for some $\varepsilon>0$, then $\int_S f\,dx\ge \varepsilon\,c(E_\varepsilon)>0$, a contradiction. Hence $c(E_\varepsilon)=0$ for all $\varepsilon>0$. Let $B=A\cup\bigcup_{m=1}^\infty E_{1/m}$. Then $c(B)=0$ and for $x\in S\setminus B$ we have $f(x)\ge 0$ and $f(x)\le 1/m$ for all $m$, hence $f(x)=0$.\qed


\begin{problembox}[14.13: Mean Value Theorem for Integrals]
\begin{problemstatement}
Assume that \( f \in R \) on \( S \), where \( S \) is a region in \( \mathbb{R}^n \) and \( f \) is continuous on \( S \). Prove that there exists an interior point \( x_0 \) of \( S \) such that
\[\int_S f(x) \, dx = f(x_0)c(S).\]
\end{problemstatement}
\end{problembox}

\noindent\textbf{Strategy:} Use the intermediate value property of continuous functions. Show that the average value of $f$ lies between the minimum and maximum, then use continuity to find a point where $f$ attains this average value.

\bigskip\noindent\textbf{Solution:}
Let $m=\min_{\overline S} f$ and $M=\max_{\overline S} f$ (attained by continuity). Then $m\,c(S)\le \int_S f\,dx\le M\,c(S)$. Choose $x_-,x_+\in S$ with $f(x_-)\le \frac{1}{c(S)}\int_S f\,dx\le f(x_+)$ (possible since $f(S)$ is an interval on each path-connected component and $S$ has nonempty interior). By continuity, along a path in $S$ joining $x_-$ to $x_+$, the intermediate value $\frac{1}{c(S)}\int_S f$ is assumed at some interior point $x_0$.\qed


\begin{problembox}[14.14: Mixed Partial Derivatives]
\begin{problemstatement}
Let \( f \) be continuous on a rectangle \( Q = [a, b] \times [c, d] \). For each interior point \( (x_1, x_2) \) in \( Q \), define
\[F(x_1, x_2) = \int_a^{x_1} \left( \int_c^{x_2} f(x, y) \, dy \right) \, dx.\]
Prove that \( D_{1,2} F(x_1, x_2) = D_{2,1} F(x_1, x_2) = f(x_1, x_2) \).
\end{problemstatement}
\end{problembox}

\noindent\textbf{Strategy:} Use the Fundamental Theorem of Calculus twice: first differentiate with respect to one variable, then the other. Show that both orders of differentiation give the same result.

\bigskip\noindent\textbf{Solution:}
By the one-variable Fundamental Theorem of Calculus and continuity of $f$,
\[ D_2 F(x_1,x_2) = \int_a^{x_1} f(x,x_2)\,dx, \qquad D_1 D_2 F(x_1,x_2) = f(x_1,x_2). \]
Symmetrically, $D_1 F(x_1,x_2)=\int_c^{x_2} f(x_1,y)\,dy$ and then $D_2 D_1 F(x_1,x_2)=f(x_1,x_2)$.\qed


\begin{problembox}[14.15: Integral of Mixed Partial Derivative]
\begin{problemstatement}
Let \( T \) denote the following triangular region in the plane:
\[T = \left\{ (x, y) : 0 \leq \frac{x}{a} + \frac{y}{b} \leq 1 \right\}, \quad \text{where } a > 0, \, b > 0.\]
Assume that \( f \) has a continuous second-order partial derivative \( D_{1,2} f \) on \( T \). Prove that there is a point \( (x_0, y_0) \) on the segment joining \( (a, 0) \) and \( (0, b) \) such that
\[\int_T D_{1,2} f(x, y) \, d(x, y) = f(0, 0) - f(a, 0) + aD_1 f(x_0, y_0).\]
\end{problemstatement}
\end{problembox}

\noindent\textbf{Strategy:} Integrate $D_{1,2}f$ first in $y$ over the triangular region, then in $x$. Use the Fundamental Theorem of Calculus and the Mean Value Theorem to find the required point.

\bigskip\noindent\textbf{Solution:}
Integrate $D_{1,2}f$ first in $y$ over $[0,\,b\,(1-\tfrac{x}{a})]$:
\[ \int_0^{b(1-x/a)} D_{1,2}f(x,y)\,dy = D_1 f\big(x, b(1-\tfrac{x}{a})\big) - D_1 f(x,0). \]
Integrating in $x\in[0,a]$ gives
\begin{align*}
 \int_T D_{1,2}f &= \int_0^a D_1 f\big(x, b(1-\tfrac{x}{a})\big)\,dx - \int_0^a D_1 f(x,0)\,dx \\ 
 &= f(0,0)-f(a,0) + \int_0^a D_1 f\big(x, b(1-\tfrac{x}{a})\big)\,dx. 
\end{align*}
By the one-dimensional Mean Value Theorem, there exists $\xi\in(0,a)$ such that the last integral equals $a\,D_1 f\big(\xi, b(1-\tfrac{\xi}{a})\big)$. This point lies on the segment joining $(a,0)$ and $(0,b)$, completing the proof.\qed

\begin{techniquessection}[Solving and Proving Techniques]

\subsection*{Working with Multiple Integrals}
\begin{itemize}
\item Use the product formula for separable integrands: $\int_S f_1(x_1) \cdots f_n(x_n) \, d(x_1, \ldots, x_n) = \prod_{i=1}^n \int_{a_i}^{b_i} f_i(x_i) \, dx_i$
\item Apply Fubini's theorem to change the order of integration
\item Use the fact that functions monotone in each variable separately are Riemann integrable
\item Apply the fact that continuous functions are Riemann integrable on compact rectangles
\item Use the fact that iterated integrals agree with multiple integrals for continuous functions
\end{itemize}

\subsection*{Proving Riemann Integrability}
\begin{itemize}
\item Use the integrability criterion via vanishing total oscillation
\item Show that the function is continuous almost everywhere
\item Apply the fact that bounded functions with content-zero discontinuities are integrable
\item Use the fact that monotone functions in each variable are integrable
\item Apply the fact that continuous functions on compact sets are integrable
\end{itemize}

\subsection*{Evaluating Multiple Integrals}
\begin{itemize}
\item Use the product formula when the integrand factors as a product of functions of single variables
\item Apply change of variables using the Jacobian determinant
\item Use symmetry to simplify calculations
\item Apply geometric decompositions to break complex regions into simpler ones
\item Use polar, cylindrical, or spherical coordinates when appropriate
\end{itemize}

\subsection*{Working with Jordan Content}
\begin{itemize}
\item Use the fact that graphs of continuous functions have content zero
\item Apply the fact that rectifiable curves have content zero
\item Use the fact that content-zero sets don't affect integrals
\item Apply Cavalieri's principle: content equals integral of section lengths
\item Use the fact that content is additive for disjoint sets
\end{itemize}

\subsection*{Applying the Mean Value Theorem}
\begin{itemize}
\item Use the fact that continuous functions attain their average value at some point
\item Apply the fact that the average value lies between the minimum and maximum
\item Use the fact that the Mean Value Theorem can be applied along paths in connected regions
\item Apply the fact that the Mean Value Theorem can be used to bound integrals
\item Use the fact that the Mean Value Theorem can be used to find points where functions attain specific values
\end{itemize}

\subsection*{Working with Mixed Partial Derivatives}
\begin{itemize}
\item Use the fact that mixed partials are equal under continuity (Clairaut's theorem)
\item Apply the Fundamental Theorem of Calculus to differentiate integrals
\item Use the fact that the order of differentiation can be interchanged under continuity
\item Apply the fact that mixed partials can be computed by iterated integration
\item Use the fact that mixed partials can be used to solve differential equations
\end{itemize}
\end{techniquessection}